% Copyright 2012 David W. Hogg (NYU).  All rights reserved.

\documentclass[pdftex]{beamer}
\setlength{\paperheight}{2.6in}
% 1.77778 is the ratio of 16 to 9
\setlength{\paperwidth}{1.77778\paperheight}
\setlength{\textwidth}{0.85\paperwidth}
\setlength{\textheight}{0.95\paperheight}
\usepackage{amssymb,amsmath,mathrsfs}
\usecolortheme{default}

% this one is debatable
\renewcommand{\emph}[1]{\textbf{#1}}

%%% color commands
\newcommand{\whiteonblack}{%
  \colorlet{fg}{white}
  \colorlet{bg}{black}
  \setbeamercolor{normal_text}{fg=white,bg=black}
  \setbeamercolor{background canvas}{fg=white,bg=black}
  \setbeamercolor{alerted_text}{fg=yellow}
  \setbeamercolor{example_text}{fg=white}
  \setbeamercolor{structure}{fg=white}
  \setbeamercolor{palette_quaternary}{fg=white}
}
\newcommand{\blackonwhite}{%
  \colorlet{fg}{black}
  \colorlet{bg}{white}
  \setbeamercolor{normal_text}{fg=black,bg=white}
  \setbeamercolor{background canvas}{fg=black,bg=white}
  \setbeamercolor{alerted_text}{fg=blue}
  \setbeamercolor{example_text}{fg=black}
  \setbeamercolor{structure}{fg=black}
  \setbeamercolor{palette_quaternary}{fg=black}
}
\xdefinecolor{pink}{rgb}{1.0,0.9,0.9}

%%% size and shape commands
\newlength{\figurewidth}
\setlength{\figurewidth}{0.9\textwidth}
\newlength{\figureheight}
\setlength{\figureheight}{0.9\textheight}

%%% text commands
\newcommand{\project}[1]{\textsl{#1}}
  \newcommand{\an}{\project{Astrometry.net}}
  \newcommand{\tc}{\project{The~Cannon}}
  \newcommand{\euclid}{\project{Euclid}}
  \newcommand{\flickr}{\project{flickr}}
  \newcommand{\gaia}{\project{Gaia}}
  \newcommand{\galex}{\project{GALEX}}
  \newcommand{\kepler}{\project{Kepler}}
  \newcommand{\GALEX}{\galex}
  \newcommand{\hst}{\project{HST}}
  \newcommand{\hipparcos}{\project{Hipparcos}}
  \newcommand{\lsst}{\project{LSST}}
  \newcommand{\sdss}{\project{SDSS}}
  \newcommand{\sdssiii}{\project{SDSS-III}}
  \newcommand{\sdssiv}{\project{SDSS-IV}}
  \newcommand{\boss}{\sdssiii\ \project{BOSS}}
  \newcommand{\osss}{\project{OSSS}}
  \newcommand{\ska}{\project{SKA}}
  \newcommand{\vo}{\project{VO}}
  \newcommand{\rttd}{\project{Right Thing To Do}$^{\mbox{\scriptsize\sffamily{TM}}}$}
\newcommand{\foreign}[1]{\textit{#1}}
\newcommand{\latin}[1]{\foreign{#1}}
  \newcommand{\cf}{\latin{cf.}}
  \newcommand{\eg}{\latin{e.g.}}
  \newcommand{\etal}{\latin{et~al.}}
  \newcommand{\etc}{\latin{etc.}}
  \newcommand{\ie}{\latin{i.e.}}
  \newcommand{\vs}{\latin{vs.}}

%%% math-mode commands
\newcommand{\unit}[1]{\mathrm{#1}}
  \newcommand{\rad}{\unit{rad}}
  \newcommand{\s}{\unit{s}}
  \newcommand{\yr}{\unit{yr}}
  \newcommand{\km}{\unit{km}}
  \newcommand{\kmps}{\km\,\s^{-1}}
\newcommand{\mmatrix}[1]{\boldsymbol{#1}}
\newcommand{\tv}[1]{\boldsymbol{#1}}
\newcommand{\dd}{\mathrm{d}}
\newcommand{\given}{\,|\,}
\newcommand{\Teff}{T_{\mathrm{eff}}}
\newcommand{\logg}{\log g}
\newcommand{\vsini}{v\,\sin i}
 % hogg standard colors
\usepackage{amssymb,amsmath,mathrsfs}

\begin{document}

\begin{frame}
  \frametitle{What is astrophysics?}
  \begin{itemize}
  \item \emph{confrontation of data or observations with theory or a model}
  \item These days, all our theories are computational.
  \end{itemize}
\end{frame}

\begin{frame}
  \frametitle{The paradox of data analysis}
  \begin{itemize}
  \item We compute the hell out of many problems:
    \begin{itemize}
    \item large-scale structure
    \item galaxies
    \item stars
    \end{itemize}
  \item But we often compare these to the data completely \emph{heuristically!}
    \begin{itemize}
    \item presume certain statistics are sufficient, or
    \item try to get the distributions right but not the individuals, or
    \item make up a wrong or substitute likelihood function
    \end{itemize}
  \item And yet \emph{we know how to propagate information}.
  \end{itemize}
\end{frame}

\begin{frame}
  \frametitle{Saturating information bounds}
  \begin{itemize}
  \item In many cases we can compute how well---in principle---we ought to be able to
    measure some quantity, given the data we have.
  \item To saturate bounds, we require a likelihood:
    \begin{itemize}
    \item to build efficient estimators
    \item to update beliefs
    \end{itemize}
  \item Or we must find a way to obtain a posterior pdf:
    \begin{itemize}
    \item (there are methods that don't require an explicit likelihood function)
    \end{itemize}
  \end{itemize}
\end{frame}

\begin{frame}
  \frametitle{Cosmology}
  \begin{itemize}
  \item (eg,) The standard likelihood function utilized in large-scale
    structure studies is \emph{not even an approximation} to any justifiable
    likelihood function.
    \begin{itemize}
    \item it has support in non-positive-definite realms
    \item the likelihood for the variance of a Gaussian isn't itself Gaussian!
    \end{itemize}
  \item New approaches built on direct simulation of the volume look promising.
    \begin{itemize}
    \item might require running (literally) millions of simulations.
    \end{itemize}
  \end{itemize}
\end{frame}

\begin{frame}
  \frametitle{The Milky Way}
  \begin{itemize}
  \item Actions and angles are beautiful things.
    \begin{itemize}
    \item the Milky Way can't be integrable
    \item observational noise must be marginalized out
    \end{itemize}
  \item The Milky Way has to form from sensible initial conditions
    within the physical cosmological model.
    \begin{itemize}
    \item at near-future observational capabilities, this \emph{one galaxy
      alone} may provide more cosmological and dark-matter constraints than the entire
      rest of the Hubble Volume
    \item exploitation probably requires something like ``constrained
      realization'' on steroids.
    \end{itemize}
  \end{itemize}
\end{frame}

\begin{frame}
  \frametitle{Stars}
  \begin{itemize}
  \item Soon we will have $10^6$ stars, each observed at $S/N\sim 100$
    in each of $10^4$ pixels (or time points).
    \begin{itemize}
    \item and yet we can't get parameters and abundances better than 0.1 or 0.2~dex!
    \end{itemize}
  \item Physical models are exceedingly computationally expensive.
    \begin{itemize}
    \item we are developing hybrid data-driven and physics-driven models
    \item the \emph{computer itself} must elaborate and refine the
      physical inputs and approximations in the models
    \end{itemize}
  \end{itemize}
\end{frame}

\begin{frame}
  \frametitle{Exoplanets}
  \begin{itemize}
  \item We know how to find exoplanets.
    \begin{itemize}
    \item we have found thousands
    \item many systems have complex architectures
    \end{itemize}
  \item We don't even know how to \emph{parameterize} the systems.
    \begin{itemize}
    \item how do you construct a pdf over qualitatively different
      kinds of systems?
    \item systems differ in multiplicity of the host star and of the
      planetary system, and of the hierarchical structure
    \item this is a problem for theory too!
    \end{itemize}
  \end{itemize}
\end{frame}

\begin{frame}
  \frametitle{The role of a Simons Center}
  \begin{itemize}
  \item Data analysis---at its best---is \emph{theory meets hardware}
  \item There is a huge role for applied mathematics.
    \begin{itemize}
    \item all futures involve simulations \emph{inside} the
      data-analysis loop
    \end{itemize}
  \end{itemize}
\end{frame}

\begin{frame}
  \frametitle{The role of a Simons Center}
  \begin{itemize}
  \item Simons could support personnel that aren't well supported in the University:
    \begin{itemize}
    \item applied math and software engineering support
    \item tool-building and maintenance groups
    \item no need to respect traditional academic silos
    \end{itemize}
  \item Students play critical roles.
    \begin{itemize}
    \item the Center could partner with a University to bring in
      PhD candidates and projects
    \item or run a predoctoral program for year-length student collaborations
    \end{itemize}
  \end{itemize}
\end{frame}

\begin{frame}
  \frametitle{The role of a Simons Center}
  \begin{itemize}
  \item Partnership in big projects is a good idea.
    \begin{itemize}
    \item the biggest data analysis challenges are brought by the biggest projects
    \item direct involvement ensures that Center accomplishments feed new discoveries
    \item best data analyses are usually performed by those with the dirtiest hands
    \end{itemize}
  \item Simons can participate in projects with more flexibility about
    time-scales than other players.
  \end{itemize}
\end{frame}

\end{document}
