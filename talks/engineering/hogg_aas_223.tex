% Copyright 2014 David W. Hogg (NYU).  All rights reserved.

\documentclass[pdftex]{beamer}
\usepackage{amssymb,amsmath,mathrsfs}
\usecolortheme{default}

% this one is debatable
\renewcommand{\emph}[1]{\textbf{#1}}

%%% color commands
\newcommand{\whiteonblack}{%
  \colorlet{fg}{white}
  \colorlet{bg}{black}
  \setbeamercolor{normal_text}{fg=white,bg=black}
  \setbeamercolor{background canvas}{fg=white,bg=black}
  \setbeamercolor{alerted_text}{fg=yellow}
  \setbeamercolor{example_text}{fg=white}
  \setbeamercolor{structure}{fg=white}
  \setbeamercolor{palette_quaternary}{fg=white}
}
\newcommand{\blackonwhite}{%
  \colorlet{fg}{black}
  \colorlet{bg}{white}
  \setbeamercolor{normal_text}{fg=black,bg=white}
  \setbeamercolor{background canvas}{fg=black,bg=white}
  \setbeamercolor{alerted_text}{fg=blue}
  \setbeamercolor{example_text}{fg=black}
  \setbeamercolor{structure}{fg=black}
  \setbeamercolor{palette_quaternary}{fg=black}
}
\xdefinecolor{pink}{rgb}{1.0,0.9,0.9}

%%% size and shape commands
\newlength{\figurewidth}
\setlength{\figurewidth}{0.9\textwidth}
\newlength{\figureheight}
\setlength{\figureheight}{0.9\textheight}

%%% text commands
\newcommand{\project}[1]{\textsl{#1}}
  \newcommand{\an}{\project{Astrometry.net}}
  \newcommand{\tc}{\project{The~Cannon}}
  \newcommand{\euclid}{\project{Euclid}}
  \newcommand{\flickr}{\project{flickr}}
  \newcommand{\gaia}{\project{Gaia}}
  \newcommand{\galex}{\project{GALEX}}
  \newcommand{\kepler}{\project{Kepler}}
  \newcommand{\GALEX}{\galex}
  \newcommand{\hst}{\project{HST}}
  \newcommand{\hipparcos}{\project{Hipparcos}}
  \newcommand{\lsst}{\project{LSST}}
  \newcommand{\sdss}{\project{SDSS}}
  \newcommand{\sdssiii}{\project{SDSS-III}}
  \newcommand{\sdssiv}{\project{SDSS-IV}}
  \newcommand{\boss}{\sdssiii\ \project{BOSS}}
  \newcommand{\osss}{\project{OSSS}}
  \newcommand{\ska}{\project{SKA}}
  \newcommand{\vo}{\project{VO}}
  \newcommand{\rttd}{\project{Right Thing To Do}$^{\mbox{\scriptsize\sffamily{TM}}}$}
\newcommand{\foreign}[1]{\textit{#1}}
\newcommand{\latin}[1]{\foreign{#1}}
  \newcommand{\cf}{\latin{cf.}}
  \newcommand{\eg}{\latin{e.g.}}
  \newcommand{\etal}{\latin{et~al.}}
  \newcommand{\etc}{\latin{etc.}}
  \newcommand{\ie}{\latin{i.e.}}
  \newcommand{\vs}{\latin{vs.}}

%%% math-mode commands
\newcommand{\unit}[1]{\mathrm{#1}}
  \newcommand{\rad}{\unit{rad}}
  \newcommand{\s}{\unit{s}}
  \newcommand{\yr}{\unit{yr}}
  \newcommand{\km}{\unit{km}}
  \newcommand{\kmps}{\km\,\s^{-1}}
\newcommand{\mmatrix}[1]{\boldsymbol{#1}}
\newcommand{\tv}[1]{\boldsymbol{#1}}
\newcommand{\dd}{\mathrm{d}}
\newcommand{\given}{\,|\,}
\newcommand{\Teff}{T_{\mathrm{eff}}}
\newcommand{\logg}{\log g}
\newcommand{\vsini}{v\,\sin i}
 % hogg standard colors
\setlength{\paperheight}{3.5in}
% 1.77778 is the ratio of 16 to 9
% 1.33333 is the ratio of 16 to 9
\setlength{\paperwidth}{1.33333\paperheight}
\setlength{\textwidth}{0.85\paperwidth}
\usepackage{amssymb,amsmath,mathrsfs}

\title{Engineering considerations\\ for large astrophysics projects}
\author[David W. Hogg (NYU)]{David W. Hogg \\[1ex]
  \textsl{\small Center for Cosmology and Particle Physics\\
                 Department of Physics\\
                 New York University\\[1ex]
                 Max-Planck-Institut f\"ur Astronomie\\
                 Heidelberg, Germany}}
\date{2014 January 9}

\newcommand{\conclusionslide}{
\begin{frame}
  \frametitle{punchlines}
  \begin{itemize}
  \item Calibration programs are wasteful and \emph{reduce the accuracy} of your end-of-mission results.
    \begin{itemize}
    \item (you will need to adjust your observing strategy)
    \end{itemize}
  \item Homogeneity and uniformity of survey samples are \emph{impossible}, \emph{unnecessary}, and \emph{harmful} goals.
    \begin{itemize}
    \item (you will need to implement some probability theory)
    \end{itemize}
  \item Proper uncertainty propagation \emph{is not easy}.
    \begin{itemize}
    \item (I got nothing)
    \end{itemize}
  \item The challenge is to make precise measurements \emph{and} keep \emph{discovery space} open.
    \begin{itemize}
    \item (you will need to understand, quantitatively, your goals)
    \end{itemize}
  \end{itemize}
\end{frame}
}

\begin{document}

\begin{frame}
  \titlepage
\end{frame}

\conclusionslide

\begin{frame}
  \frametitle{my teachers (incomplete list)}
  \begin{itemize}
  \item Gerry Neugebauer (Caltech, emeritus)
  \item Sam Roweis (Toronto \& NYU, deceased)
  \item Dave~Schlegel~(LBL) \& Scott~Burles~(Cutler)
  \item Mike Blanton (NYU)
  \item Dustin~Lang~(CMU) \& Jo~Bovy~(IAS) \& Dan~Foreman-Mackey~(NYU)
  \end{itemize}
\end{frame}

\begin{frame}
  \frametitle{survey-centric context}
  \begin{itemize}
  \item \gaia
  \item \ska\ and pathfinders
  \item \euclid
  \item \lsst
  \item \sdssiv
    \begin{itemize}
    \item (I am going to get mean at the end.)
    \end{itemize}
  \end{itemize}
\end{frame}

\begin{frame}
  \frametitle{homogeneity and uniformity are impossible}
  \begin{itemize}
  \item weather
  \item target selection
  \item hardware evolution
  \item efficiency considerations
  \end{itemize}
\end{frame}

\begin{frame}
  \frametitle{probabilistic target selection}
  \begin{itemize}
  \item \sdssiii, \sdssiv
  \end{itemize}
\end{frame}

\begin{frame}
  \frametitle{homogeneity and uniformity are unnecessary}
  \begin{itemize}
  \item $1 / V_{\mathrm{max}}$
  \item forward modeling
  \item visualizing a forward model
  \end{itemize}
\end{frame}

\begin{frame}
  \frametitle{estimators}
  \begin{itemize}
  \item likelihood principles
  \item Cram\`er--Rao bound
  \item it is \emph{our duty} to analyze our \emph{very limited data} with optimal methods
  \end{itemize}
\end{frame}

\begin{frame}
  \frametitle{target-selection pitfalls}
  \begin{itemize}
  \item SEGUE example
  \end{itemize}
\end{frame}

\begin{frame}
  \frametitle{homogeneity and uniformity are unnecessary?}
  \begin{itemize}
  \item special case of two-point functions (and higher orders)
  \end{itemize}
\end{frame}

\begin{frame}
  \frametitle{homogeneity and uniformity are harmful}
  \begin{itemize}
  \item dynamic range issues
  \item can't be uniform in \emph{everything}
  \item uniform samples end up requiring a lot of time on the least useful objects
  \item reduces heterogeneity that is essential to calibration
  \end{itemize}
\end{frame}

\begin{frame}
  \frametitle{self-calibration}
  \begin{itemize}
  \item SDSS
  \item Euclid
  \end{itemize}
\end{frame}

\begin{frame}
  \frametitle{calibration programs are wasteful}
  \begin{itemize}
  \item foo
  \item bar
  \end{itemize}
\end{frame}

\begin{frame}
  \frametitle{target selection is classification}
  \begin{itemize}
  \item \sdssiii\ \boss\ is taking spectra of quasars, not stars
  \item stars outnumber (relevant) quasars by factors of hundreds
  \item observations are noisy and theoretical models are incomplete
  \item want to find \emph{only} the quasars\ldots or do we?
  \end{itemize}
\end{frame}

\begin{frame}
  \frametitle{classification algorithms}
  \begin{itemize}
  \item SupportVectorMachine, RandomForest, ArtificialNeuralNet
  \item value of a \emph{causal} model
    \begin{itemize}
    \item Training and test samples don't match.
    \item Need to classify new data taken under \emph{different conditions}.
    \end{itemize}
  \end{itemize}
\end{frame}

\begin{frame}
  \frametitle{aside: discovery as classification}
  \begin{itemize}
  \item found an exoplanet?
    \begin{itemize}
    \item That's a model selection move.
    \item Bayes doesn't tell you how to \emph{make decisions}.
    \end{itemize}
  \item utility arises
    \begin{itemize}
    \item Make decisions that maximize expected (scientific?) return.
    \end{itemize}
  \item \an\ (Lang \etal) has an explicit utility model
    \begin{itemize}
    \item Automatic calibration of an image successful?
    \item Our ``customer model'' is that they are offended by false positives.
    \end{itemize}
  \end{itemize}
\end{frame}

\begin{frame}
  \frametitle{utility considerations}
  \begin{itemize}
  \item long-term future discounted free-cash flow
  \item might be worth taking a source unlikely to be a quasar, but if it is a star, it is an \emph{interesting} star
  \end{itemize}
\end{frame}

\begin{frame}
  \frametitle{over-design}
  \begin{itemize}
  \item \sdss\ was seriously over-designed to measure the large-scale structure
    \begin{itemize}
    \item (no-one thinks that was a bad idea)
    \item we might have to be more honest going forward
    \end{itemize}
  \item if we want to use resources efficiently, we need to face the trade-off between efficiency and discovery
  \end{itemize}
\end{frame}

\begin{frame}
  \frametitle{every choice can be turned into an optimization}
  \begin{itemize}
  \item foo
  \item bar
  \end{itemize}
\end{frame}

\begin{frame}
  \frametitle{example: bandpasses}
  \begin{itemize}
  \item foo
  \item bar
  \end{itemize}
\end{frame}

\begin{frame}
  \frametitle{propagating uncertainty}
  \begin{itemize}
  \item foo
  \item bar
  \end{itemize}
\end{frame}

\begin{frame}
  \frametitle{probabilistic outputs}
  \begin{itemize}
  \item foo
  \item bar
  \end{itemize}
\end{frame}

\begin{frame}
  \frametitle{hardware \foreign{vs} software}
  \begin{itemize}
  \item P1640
  \item Kepler
  \item Spitzer
  \end{itemize}
\end{frame}

\begin{frame}
  \frametitle{open-source surveys}
  \begin{itemize}
  \item \hipparcos\ example
  \end{itemize}
\end{frame}

\begin{frame}
  \frametitle{throwing down the gauntlet}
  \begin{itemize}
  \item \gaia\ uncertainty propagation (qualitative)
  \item \euclid\ observing strategy for imaging
  \item \lsst\ bandpass, cadence, and exposure-time settings
  \item \project{eBOSS} two-point function estimators
  \item \project{APOGEE} \& \project{HERMES} signal-to-noise requirements
    \begin{itemize}
    \item (My hourly rates are a bargain.)
    \item (These surveys are all \emph{awesome}!)
    \end{itemize}
  \end{itemize}
\end{frame}

\conclusionslide

\end{document}
