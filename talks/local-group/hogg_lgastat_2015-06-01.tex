% Copyright 2012 David W. Hogg (NYU).  All rights reserved.

\documentclass[pdftex]{beamer}
\usepackage{amssymb,amsmath,mathrsfs}
\usecolortheme{default}

% this one is debatable
\renewcommand{\emph}[1]{\textbf{#1}}

%%% color commands
\newcommand{\whiteonblack}{%
  \colorlet{fg}{white}
  \colorlet{bg}{black}
  \setbeamercolor{normal_text}{fg=white,bg=black}
  \setbeamercolor{background canvas}{fg=white,bg=black}
  \setbeamercolor{alerted_text}{fg=yellow}
  \setbeamercolor{example_text}{fg=white}
  \setbeamercolor{structure}{fg=white}
  \setbeamercolor{palette_quaternary}{fg=white}
}
\newcommand{\blackonwhite}{%
  \colorlet{fg}{black}
  \colorlet{bg}{white}
  \setbeamercolor{normal_text}{fg=black,bg=white}
  \setbeamercolor{background canvas}{fg=black,bg=white}
  \setbeamercolor{alerted_text}{fg=blue}
  \setbeamercolor{example_text}{fg=black}
  \setbeamercolor{structure}{fg=black}
  \setbeamercolor{palette_quaternary}{fg=black}
}
\xdefinecolor{pink}{rgb}{1.0,0.9,0.9}

%%% size and shape commands
\newlength{\figurewidth}
\setlength{\figurewidth}{0.9\textwidth}
\newlength{\figureheight}
\setlength{\figureheight}{0.9\textheight}

%%% text commands
\newcommand{\project}[1]{\textsl{#1}}
  \newcommand{\an}{\project{Astrometry.net}}
  \newcommand{\tc}{\project{The~Cannon}}
  \newcommand{\euclid}{\project{Euclid}}
  \newcommand{\flickr}{\project{flickr}}
  \newcommand{\gaia}{\project{Gaia}}
  \newcommand{\galex}{\project{GALEX}}
  \newcommand{\kepler}{\project{Kepler}}
  \newcommand{\GALEX}{\galex}
  \newcommand{\hst}{\project{HST}}
  \newcommand{\hipparcos}{\project{Hipparcos}}
  \newcommand{\lsst}{\project{LSST}}
  \newcommand{\sdss}{\project{SDSS}}
  \newcommand{\sdssiii}{\project{SDSS-III}}
  \newcommand{\sdssiv}{\project{SDSS-IV}}
  \newcommand{\boss}{\sdssiii\ \project{BOSS}}
  \newcommand{\osss}{\project{OSSS}}
  \newcommand{\ska}{\project{SKA}}
  \newcommand{\vo}{\project{VO}}
  \newcommand{\rttd}{\project{Right Thing To Do}$^{\mbox{\scriptsize\sffamily{TM}}}$}
\newcommand{\foreign}[1]{\textit{#1}}
\newcommand{\latin}[1]{\foreign{#1}}
  \newcommand{\cf}{\latin{cf.}}
  \newcommand{\eg}{\latin{e.g.}}
  \newcommand{\etal}{\latin{et~al.}}
  \newcommand{\etc}{\latin{etc.}}
  \newcommand{\ie}{\latin{i.e.}}
  \newcommand{\vs}{\latin{vs.}}

%%% math-mode commands
\newcommand{\unit}[1]{\mathrm{#1}}
  \newcommand{\rad}{\unit{rad}}
  \newcommand{\s}{\unit{s}}
  \newcommand{\yr}{\unit{yr}}
  \newcommand{\km}{\unit{km}}
  \newcommand{\kmps}{\km\,\s^{-1}}
\newcommand{\mmatrix}[1]{\boldsymbol{#1}}
\newcommand{\tv}[1]{\boldsymbol{#1}}
\newcommand{\dd}{\mathrm{d}}
\newcommand{\given}{\,|\,}
\newcommand{\Teff}{T_{\mathrm{eff}}}
\newcommand{\logg}{\log g}
\newcommand{\vsini}{v\,\sin i}
 % hogg standard colors
\setlength{\paperheight}{3.5in}
% 1.77778 is the ratio of 16 to 9
\setlength{\paperwidth}{1.77778\paperheight}
\setlength{\textwidth}{0.85\paperwidth}
\usepackage{amssymb,amsmath,mathrsfs}

\title{Mapping the dark matter and inferring the formation history of the Local Group}
\author[David W. Hogg (NYU)]{David W. Hogg \\
  \textsl{\footnotesize Center for Cosmology and Particle Physics, Department of Physics,
                 New York University}\\
  \textsl{\footnotesize Center for Data Science,
                 New York University}\\
  \textsl{\footnotesize Max-Planck-Institut f\"ur Astronomie}}
\date{2015 June 01}

\begin{document}

\begin{frame}
  \titlepage
\end{frame}

\newcommand{\messages}{%
\begin{frame}
  \frametitle{messages}
  \begin{itemize}
  \item Naively, any inference of the potential $\Phi$ given
    phase-space coordinates $X_n$ of stars $n$ requires \emph{enormous
      marginalizations} (or profiles).
    \begin{itemize}
    \item By ``potential'' I mean all dynamical properties as a function of space and time.
    \item There can be a \emph{tiny subset} of the stars that is most informative.
    \end{itemize}
  \item Extending $X_n$ to include more conserved ``tags'' is expected to be informative
    \begin{itemize}
    \item (But vastly increases the number of nuisances.)
    \end{itemize}
  \item In 2023, if all we have is a \emph{very precise mass} for
    each Local-Group member, we will have \emph{failed miserably}.
    \begin{itemize}
    \item This is not like cosmological parameter estimation!
    \item Better data must bring qualitatively new insights.
    \end{itemize}
  \item All projects so far have just been \emph{toys}.
  \end{itemize}
\end{frame}}

\messages

\begin{frame}
  \frametitle{challenges}
  \begin{itemize}
  \item our prior beliefs are generated by \emph{non-linear physical cosmology}
    \begin{itemize}
    \item (plus small modifications and adjustments)
    \item physical or effective models of star formation?
    \item intractable likelihood function?
    \end{itemize}
  \item data sets are vast and \emph{noisy}
    \begin{itemize}
    \item delivering probabilistic outputs from telescopes
    \item writing inferences with posteriors in and posteriors out
    \item non-parametric models get huge ($N^{2.6}$-ish)
    \end{itemize}
  \item \emph{stars} are non-trivial objects
    \begin{itemize}
    \item consistent parameters across surveys
    \item consistent chemical abundances across $\Teff$, $\logg$, $\vsini$, \etc
    \end{itemize}
  \end{itemize}
\end{frame}

\begin{frame}
  \frametitle{dynamical inference}
  \begin{itemize}
  \item The positions and velocities $X_n$ are \emph{initial conditions}.
  \item The potential $\Phi$ and its history describe the dynamics.
  \item \emph{You can't infer one from the other.}
    \begin{itemize}
    \item (\cf, undergraduate dynamics)
    \item<2-> but what if we make other assumptions?
    \end{itemize}
  \item<2-> The Milky Way is long-lived.
    \begin{itemize}
    \item and we aren't seeing it at a special time
    \item and there are no significant resonances (stars are independent)
    \item (these assumptions are all wrong in detail)
    \end{itemize}
  \item<2-> Then we know what to do!
  \end{itemize}
\end{frame}

\begin{frame}
  \frametitle{April Fools' {\footnotesize Bovy \etal, 0903.5308}}
  \begin{itemize}
  \item Can we learn the gravitational force law in the Solar System
    given \emph{only a snapshot} of the planet phase-space positions?
  \item Rough idea:
    \begin{itemize}
    \item For a long-lived, integrable, non-resonant system, angles
      are uniformly distributed, actions are conserved.
    \item The action distribution matters and is completely unknown.
    \item Model needs flexibility to discover near-circularity of the
      orbits.
    \item Many assumptions (sperical, power-law, time-invariant, integrable)!
    \end{itemize}
  \item $\displaystyle p(x,v\given\Phi) = \int p(x,v\given I,\theta,\Phi)\,p(I\given\alpha)\,p(\theta)\,\dd I\,\dd\theta\,\dd\alpha$
  \item Jacobians of transformations from $x,v$ to $I,\theta$
    coordinates describe the information in the data.
  \end{itemize}
\end{frame}

\begin{frame}
  \frametitle{April Fools' {\footnotesize Bovy \etal, 0903.5308}}
  ~\hfill%
  \includegraphics<1>[height=\figureheight]{0903.5308/f5a.pdf}
  \includegraphics<2>[height=\figureheight]{0903.5308/f5b.pdf}
  \includegraphics<3>[height=\figureheight]{0903.5308/f5c.pdf}
  \includegraphics<4>[height=\figureheight]{0903.5308/f5d.pdf}
  \includegraphics<5>[height=\figureheight]{0903.5308/f5e.pdf}
  \includegraphics<6>[height=\figureheight]{0903.5308/f5f.pdf}
  \includegraphics<7>[height=\figureheight]{0903.5308/f5g.pdf}
  \includegraphics<8>[height=\figureheight]{0903.5308/f5h.pdf}
  \includegraphics<9>[height=\figureheight]{0903.5308/f7.pdf}
\end{frame}

\begin{frame}
  \frametitle{April Fools' {\footnotesize Bovy \etal, 0903.5308}}
  \begin{itemize}
  \item Inference of the potential $\Phi$ given phase-space postions
    $X_n$ is possible.
  \item Marginalization was incredibly \emph{expensive}.
    \begin{itemize}
    \item CPU-days for $N=8$ particles!
    \end{itemize}
  \item Some ``stars'' were \emph{far} more valuable than others.
  \item Lessons for 2023?
    \begin{itemize}
    \item Computational expense daunting.
    \item Key stars are worth finding (if they exist).
    \item Things won't be integrable, stationary, or symmetric.
    \end{itemize}
  \end{itemize}
\end{frame}

\begin{frame}
  \frametitle{measuring chemical abundances}
  \begin{itemize}
  \item there is \emph{no hope} of good, consistent point-estimates of
    chemical abundances
    \begin{itemize}
    \item (not for 2023, anyway)
    \item prove me wrong!
    \item (and keep working; this is God's work)
    \end{itemize}
  \item we have to see the chemical tagging as being done
    \emph{simultaneously} with the dynamical fitting.
    \begin{itemize}
    \item either effective models of stellar spectra, or else
    \item parameterizing all stellar model uncertainties and treating them as nuisances
    \item (these models will get enormous)
    \end{itemize}
  \end{itemize}
\end{frame}

\begin{frame}
  \frametitle{\tc~{\footnotesize Ness \etal, 1501.07604}}
  \begin{itemize}
  \item foo
    \begin{itemize}
    \item (plus small modifications and adjustments)
    \item physical or effective models of star formation?
    \item intractable likelihood function?
    \end{itemize}
  \item bar
    \begin{itemize}
    \item delivering probabilistic outputs from telescopes
    \item writing inferences with posteriors in and posteriors out
    \item non-parametric models get huge ($N^3$-ish)
    \end{itemize}
  \end{itemize}
\end{frame}

\begin{frame}
  \frametitle{\tc~{\footnotesize Ness \etal, 1501.07604}}
  \begin{itemize}
  \item foo
    \begin{itemize}
    \item (plus small modifications and adjustments)
    \item physical or effective models of star formation?
    \item intractable likelihood function?
    \end{itemize}
  \item bar
    \begin{itemize}
    \item delivering probabilistic outputs from telescopes
    \item writing inferences with posteriors in and posteriors out
    \item non-parametric models get huge ($N^3$-ish)
    \end{itemize}
  \end{itemize}
\end{frame}

\messages

\end{document}
