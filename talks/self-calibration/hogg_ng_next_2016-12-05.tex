% Copyright 2016 David W. Hogg (NYU).  All rights reserved.

\documentclass[pdftex]{beamer}
\usepackage{amssymb,amsmath,mathrsfs}
\usecolortheme{default}

% this one is debatable
\renewcommand{\emph}[1]{\textbf{#1}}

%%% color commands
\newcommand{\whiteonblack}{%
  \colorlet{fg}{white}
  \colorlet{bg}{black}
  \setbeamercolor{normal_text}{fg=white,bg=black}
  \setbeamercolor{background canvas}{fg=white,bg=black}
  \setbeamercolor{alerted_text}{fg=yellow}
  \setbeamercolor{example_text}{fg=white}
  \setbeamercolor{structure}{fg=white}
  \setbeamercolor{palette_quaternary}{fg=white}
}
\newcommand{\blackonwhite}{%
  \colorlet{fg}{black}
  \colorlet{bg}{white}
  \setbeamercolor{normal_text}{fg=black,bg=white}
  \setbeamercolor{background canvas}{fg=black,bg=white}
  \setbeamercolor{alerted_text}{fg=blue}
  \setbeamercolor{example_text}{fg=black}
  \setbeamercolor{structure}{fg=black}
  \setbeamercolor{palette_quaternary}{fg=black}
}
\xdefinecolor{pink}{rgb}{1.0,0.9,0.9}

%%% size and shape commands
\newlength{\figurewidth}
\setlength{\figurewidth}{0.9\textwidth}
\newlength{\figureheight}
\setlength{\figureheight}{0.9\textheight}

%%% text commands
\newcommand{\project}[1]{\textsl{#1}}
  \newcommand{\an}{\project{Astrometry.net}}
  \newcommand{\tc}{\project{The~Cannon}}
  \newcommand{\euclid}{\project{Euclid}}
  \newcommand{\flickr}{\project{flickr}}
  \newcommand{\gaia}{\project{Gaia}}
  \newcommand{\galex}{\project{GALEX}}
  \newcommand{\kepler}{\project{Kepler}}
  \newcommand{\GALEX}{\galex}
  \newcommand{\hst}{\project{HST}}
  \newcommand{\hipparcos}{\project{Hipparcos}}
  \newcommand{\lsst}{\project{LSST}}
  \newcommand{\sdss}{\project{SDSS}}
  \newcommand{\sdssiii}{\project{SDSS-III}}
  \newcommand{\sdssiv}{\project{SDSS-IV}}
  \newcommand{\boss}{\sdssiii\ \project{BOSS}}
  \newcommand{\osss}{\project{OSSS}}
  \newcommand{\ska}{\project{SKA}}
  \newcommand{\vo}{\project{VO}}
  \newcommand{\rttd}{\project{Right Thing To Do}$^{\mbox{\scriptsize\sffamily{TM}}}$}
\newcommand{\foreign}[1]{\textit{#1}}
\newcommand{\latin}[1]{\foreign{#1}}
  \newcommand{\cf}{\latin{cf.}}
  \newcommand{\eg}{\latin{e.g.}}
  \newcommand{\etal}{\latin{et~al.}}
  \newcommand{\etc}{\latin{etc.}}
  \newcommand{\ie}{\latin{i.e.}}
  \newcommand{\vs}{\latin{vs.}}

%%% math-mode commands
\newcommand{\unit}[1]{\mathrm{#1}}
  \newcommand{\rad}{\unit{rad}}
  \newcommand{\s}{\unit{s}}
  \newcommand{\yr}{\unit{yr}}
  \newcommand{\km}{\unit{km}}
  \newcommand{\kmps}{\km\,\s^{-1}}
\newcommand{\mmatrix}[1]{\boldsymbol{#1}}
\newcommand{\tv}[1]{\boldsymbol{#1}}
\newcommand{\dd}{\mathrm{d}}
\newcommand{\given}{\,|\,}
\newcommand{\Teff}{T_{\mathrm{eff}}}
\newcommand{\logg}{\log g}
\newcommand{\vsini}{v\,\sin i}
 % hogg standard colors
\setlength{\paperheight}{3.5in}
% 1.77778 is the ratio of 16 to 9
% 1.33333 is the ratio of 16 to 9
\setlength{\paperwidth}{1.77778\paperheight}
\setlength{\textwidth}{0.85\paperwidth}
\usepackage{amssymb,amsmath,mathrsfs}

\newcommand{\data}{\mbox{data}}
\newcommand{\pars}{\mbox{parameters}}

\title{Radical ideas in self-calibration}
\author[David W. Hogg (NYU)]{David W. Hogg \\[1ex]
  \textsl{\small Center for Cosmology and Particle Physics\\
                 Department of Physics\\
                 New York University\\[1ex]
                 Center for Data Science\\
                 New York University}}
\date{NG NEXT / 2016 December 5}

\newcommand{\conclusionslide}{
\begin{frame}
  \frametitle{punchlines}
  \begin{itemize}
  \item traditional calibration programs are wasteful and \emph{reduce the accuracy} of your end-of-mission results.
  \item \emph{consistency of the data} provides an almost-complete calibration.
  \item we exploit \emph{causal structure} to generate multiple modalities of consistency.
  \end{itemize}
\end{frame}
}

\begin{document}

\begin{frame}
  \titlepage
\end{frame}

\conclusionslide

\begin{frame}
  \frametitle{Traditional calibration}
  \begin{itemize}
  \item intersperse calibration (standard-star) observations within the sequence of science observations
  \item set parameters to minimize variance across calibrator observations
  \item apply those same parameters to the science observations
  \end{itemize}
\end{frame}

\begin{frame}
  \frametitle{``Traditional'' self-calibration}
  \begin{itemize}
  \item (what we did in the \project{Sloan Digital Sky Survey})
  \item each star is observed many times, in many different detector positions
  \item set parameters to minimize covariance of stellar brightness with detector position
  \item obviates calibration standards
  \item (requires data redundancy)
  \end{itemize}
\end{frame}

\begin{frame}
  \frametitle{Advantages of self-calibration}
  \begin{itemize}
  \item reduces overheads
  \item calibration information comes from observing in science ``modes''
  \item requires no extrapolation or interpolation
  \item (works better \emph{in practice})
  \end{itemize}
\end{frame}

\begin{frame}
  \frametitle{Point-spread function}
  \begin{itemize}
  \item each detector position is illuminated many times, by many different stars
  \item set parameters to minimize covariance of data residuals with illuminating star
  \item (this has been the approach for decades)
  \end{itemize}
\end{frame}

\begin{frame}
  \frametitle{Time-domain astrophysics}
  \begin{itemize}
  \item in (for example) \textsl{Kepler}, measure brightness vs time over long times
  \item what is due to spacecraft, what is due to stellar variability?
  \item what you can predict about one star from other stars must be spacecraft
  \item (limit of large data)
  \item (nonlinear predictions)
  \end{itemize}
\end{frame}

\begin{frame}
  \frametitle{Image differencing}
  \begin{itemize}
  \item what you can predict about one pixel from other pixels must be spacecraft
  \item (microlensing discoveries)
  \end{itemize}
\end{frame}

\begin{frame}
  \frametitle{Radial-velocity measurements}
  \begin{itemize}
  \item tens to thousands of epochs of spectroscopy at $R=115,000$, $\mathrm{SNR}>100$.
  \item different exposures should differ \emph{only} by radial velocity
  \item if there are spectral shape variations with radial velocity, we can do better
  \item (note causal argument)
  \end{itemize}
\end{frame}

\begin{frame}
  \frametitle{Binary star radial-velocity measurements}
  \begin{itemize}
  \item if we can see two stars in the spectrum, each star can velocity-calibrate the other
  \item the most important exoplanets might be in binary-star systems
  \end{itemize}
\end{frame}

\begin{frame}
  \frametitle{Pixel-level flat-fields}
  \begin{itemize}
  \item telescope images are produced by a ``true scene'' convolved by the point-spread function
  \item any image can be modeled as a mixture of PSFs
  \item deviations consistent in detector coordinates must be flat-field issues
  \item (note causal argument)
  \item (doesn't require any data redundancy, in principle)
  \end{itemize}
\end{frame}

\begin{frame}
  \frametitle{Sub-pixel flat-fields}
  \begin{itemize}
  \item every pixel has a unique shape
  \item with enough data, this can be inferred also
  \item concept of a flat-field multipole expansion
  \item (much easier with certain kinds of redundancy)
  \end{itemize}
\end{frame}

\begin{frame}
  \frametitle{Photometric redshifts}
  \begin{itemize}
  \item galaxy spectra live in some nonlinear subspace
  \item every galaxy ever observed, photometrically, constrains that subspace
  \item (photometric measurements are noisy projections of the spectrum)
  \item photometric redshifts can be inferred by locating objects in the latent subspace
  \end{itemize}
\end{frame}

\begin{frame}
  \frametitle{High-contrast imaging}
  \begin{itemize}
  \item in (for example) \textsl{GPI}, null the light of the primary star
  \item extremely featured residuals, mimicking planets
  \item planets are rare; anything you have ``seen before'' isn't a planet
  \item what's been ``seen before'' lies on a nonlinear subspace of data space
  \item (spectral information is valuable)
  \end{itemize}
\end{frame}

\begin{frame}
  \frametitle{Bringing it all together}
  \begin{itemize}
  \item \textsl{JWST} and future \textsl{HabEx}-like missions will use all of these self-calibration methods
  \item traditional self-calibration for the PSF and flat-field
  \item sub-pixel flat-field will be needed for time domain
  \item data-driven models for stars and galaxies
  \item speckle modeling with subspace determination
  \end{itemize}
\end{frame}

\begin{frame}
  \frametitle{Observing strategies}
  \begin{itemize}
  \item We are going to have very strong opinions about how the data are taken!
  \item dithering and redundancy
  \item exposure times and numbers of frames
  \item heterogeneity
  \item (calibration will be bandwidth-limited)
  \end{itemize}
\end{frame}

\conclusionslide

\end{document}
