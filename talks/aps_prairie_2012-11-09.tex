% Copyright 2012 David W. Hogg (NYU).  All rights reserved.

\documentclass[pdftex]{beamer}
\usepackage{amssymb,amsmath,mathrsfs}
\usecolortheme{default}

% this one is debatable
\renewcommand{\emph}[1]{\textbf{#1}}

%%% color commands
\newcommand{\whiteonblack}{%
  \colorlet{fg}{white}
  \colorlet{bg}{black}
  \setbeamercolor{normal_text}{fg=white,bg=black}
  \setbeamercolor{background canvas}{fg=white,bg=black}
  \setbeamercolor{alerted_text}{fg=yellow}
  \setbeamercolor{example_text}{fg=white}
  \setbeamercolor{structure}{fg=white}
  \setbeamercolor{palette_quaternary}{fg=white}
}
\newcommand{\blackonwhite}{%
  \colorlet{fg}{black}
  \colorlet{bg}{white}
  \setbeamercolor{normal_text}{fg=black,bg=white}
  \setbeamercolor{background canvas}{fg=black,bg=white}
  \setbeamercolor{alerted_text}{fg=blue}
  \setbeamercolor{example_text}{fg=black}
  \setbeamercolor{structure}{fg=black}
  \setbeamercolor{palette_quaternary}{fg=black}
}
\xdefinecolor{pink}{rgb}{1.0,0.9,0.9}

%%% size and shape commands
\newlength{\figurewidth}
\setlength{\figurewidth}{0.9\textwidth}
\newlength{\figureheight}
\setlength{\figureheight}{0.9\textheight}

%%% text commands
\newcommand{\project}[1]{\textsl{#1}}
  \newcommand{\an}{\project{Astrometry.net}}
  \newcommand{\tc}{\project{The~Cannon}}
  \newcommand{\euclid}{\project{Euclid}}
  \newcommand{\flickr}{\project{flickr}}
  \newcommand{\gaia}{\project{Gaia}}
  \newcommand{\galex}{\project{GALEX}}
  \newcommand{\kepler}{\project{Kepler}}
  \newcommand{\GALEX}{\galex}
  \newcommand{\hst}{\project{HST}}
  \newcommand{\hipparcos}{\project{Hipparcos}}
  \newcommand{\lsst}{\project{LSST}}
  \newcommand{\sdss}{\project{SDSS}}
  \newcommand{\sdssiii}{\project{SDSS-III}}
  \newcommand{\sdssiv}{\project{SDSS-IV}}
  \newcommand{\boss}{\sdssiii\ \project{BOSS}}
  \newcommand{\osss}{\project{OSSS}}
  \newcommand{\ska}{\project{SKA}}
  \newcommand{\vo}{\project{VO}}
  \newcommand{\rttd}{\project{Right Thing To Do}$^{\mbox{\scriptsize\sffamily{TM}}}$}
\newcommand{\foreign}[1]{\textit{#1}}
\newcommand{\latin}[1]{\foreign{#1}}
  \newcommand{\cf}{\latin{cf.}}
  \newcommand{\eg}{\latin{e.g.}}
  \newcommand{\etal}{\latin{et~al.}}
  \newcommand{\etc}{\latin{etc.}}
  \newcommand{\ie}{\latin{i.e.}}
  \newcommand{\vs}{\latin{vs.}}

%%% math-mode commands
\newcommand{\unit}[1]{\mathrm{#1}}
  \newcommand{\rad}{\unit{rad}}
  \newcommand{\s}{\unit{s}}
  \newcommand{\yr}{\unit{yr}}
  \newcommand{\km}{\unit{km}}
  \newcommand{\kmps}{\km\,\s^{-1}}
\newcommand{\mmatrix}[1]{\boldsymbol{#1}}
\newcommand{\tv}[1]{\boldsymbol{#1}}
\newcommand{\dd}{\mathrm{d}}
\newcommand{\given}{\,|\,}
\newcommand{\Teff}{T_{\mathrm{eff}}}
\newcommand{\logg}{\log g}
\newcommand{\vsini}{v\,\sin i}
 % hogg standard colors
\setlength{\paperheight}{3.5in}
% 1.77778 is the ratio of 16 to 9
\setlength{\paperwidth}{1.77778\paperheight}
\setlength{\textwidth}{0.85\paperwidth}
\usepackage{amssymb,amsmath,mathrsfs}

\title{Big data challenges for physics in the next decades}
\author[David W. Hogg (NYU)]{David W. Hogg \\
  \textsl{\small Center for Cosmology and Particle Physics,
                 New York University}}
\date{2012 November 09}

\newcommand{\conclusion}{
\begin{frame}
  \frametitle{punchlines}
  \begin{itemize}
  \item Huge data sets create new opportunities.
    \begin{itemize}
    \item they also present new challenges
    \end{itemize}
  \item Industrial (dot-com) methods are brilliant.
    \begin{itemize}
    \item they can only solve a very limited set of problems
    \end{itemize}
  \item We won't reap the full benefit of larger data sets without new technology.
    \begin{itemize}
    \item call to arms
    \item (and get rich too!)
    \end{itemize}
  \end{itemize}
\end{frame}
}

\begin{document}

\begin{frame}
  \titlepage
\end{frame}

\conclusion

\begin{frame}
  \frametitle{My day job}
  \begin{itemize}
  \item Lang \& Hogg {\small(forthcoming)}: a $10^9$-parameter model of the
    $10^{13}$ \project{Sloan Digital Sky Survey} pixels (\project{The Tractor})
  \item Brewer \etal\ {\small(forthcoming)}: Bayesian non-parametrics but with
    priors that represent our actual prior knowledge
  \item Foreman-Mackey \etal\ {\small(arXiv:1202.3665)}:
    \project{emcee}, the MCMC Hammer: flexible, parallelized, adaptive
    sampler
  \item Bovy, Murray, Hogg {\small(arXiv:0903.5308)}: a dynamical inference
    fully marginalizing out a non-parametric distribution function
  \item Bovy \etal\ {\small(arXiv:1105.3975)}: a 60,000-parameter model of
    700,000 flux measurements, followed by predictions for 160,000,000
    point sources
  \end{itemize}
\end{frame}

\begin{frame}
  \frametitle{scale}
  \begin{itemize}
  \item \project{Large Synoptic Survey Telescope}: $10^{10}$ galaxies in $10^{15}$ pixels
    \begin{itemize}
    \item get the cosmic gravitational-lensing shear map
    \item and then the cosmological parameters
    \item $10^9$ stars too
    \end{itemize}
  \item \project{Gaia}: $10^{9}$ stars in $10^{12}$ pixels
    \begin{itemize}
    \item infer the dynamics of the Milky Way
    \item but also---necessarily---the distribution function of stars
      in that potential
    \item precision requirements are \emph{outrageous} (milli-pixel positions)
    \end{itemize}
  \item \project{Large Hadron Collider} instruments: taking data $10^7$ times faster than it can be moved to disk
    \begin{itemize}
    \item they found the Higgs! (probably)
    \item have to make hard cuts and \emph{throw away data}
    \end{itemize}
  \end{itemize}
\end{frame}

\begin{frame}
  \frametitle{how do you store big data sets?}
  \begin{itemize}
  \item Distribute the data.
    \begin{itemize}
    \item $< 1$~Tb of data per CPU, thousands of CPUs
    \item all modern databases can do this, even open-source ones
    \end{itemize}
  \item There is a CPU near every data point.
    \begin{itemize}
    \item and there are many, many CPUs
    \end{itemize}
  \item Hardware is cheap; management is expensive.
  \item Massive CPU redundancy creates opportunities\ldots
  \end{itemize}
\end{frame}

\begin{frame}
  \frametitle{map--reduce or die}
  \begin{itemize}
  \item \textsl{``We won't even consider algorithms that can't be
    written in the map--reduce framework.''}
  \item map:
    \begin{itemize}
    \item at each ``data point'' (on the distributed system), do an
      operation on that datum, produce output
    \item \texttt{if "kittens" is in document: \\ ~~~~return (URL, PageRank)}
    \item \emph{distributed data} is the key: Every datum is near a
      CPU.
    \end{itemize}
  \item reduce:
    \begin{itemize}
    \item between each pair of outputs, do an operation and return one
      new output, recurse up the tree
    \item \texttt{if PageRank[0] > PageRank[1]: \\ ~~~~return (URL[0], PageRank[0]) \\ else: \\ ~~~~return (URL[1], PageRank[1])}
    \item \emph{tree structure} of the data center is the key: There are only
      $\log_2 N$ branches to any datum.
    \end{itemize}
  \end{itemize}
\end{frame}

\begin{frame}
  \frametitle{map--reduce or die}
  \begin{itemize}
  \item Brilliant.  And a huge opportunity.
    \begin{itemize}
    \item the computational complexity is $N\,\log N$
    \item but the \emph{time it takes} scales as just $\log N$
    \end{itemize}
  \item Map--reduce made internet search \emph{possible}.
    \begin{itemize}
    \item (goes by many names, some trade-marked!)
    \end{itemize}
  \end{itemize}
\end{frame}

\begin{frame}
  \frametitle{maximum-likelihood and map--reduce}
  \begin{itemize}
  \item full-data likelihood: $\displaystyle p(D\given\theta) =
    \prod_n p(d_n\given\theta)$
  \item Find the \emph{maximum with respect to $\theta$} of this
    likelihood.
  \item map:
    \begin{itemize}
    \item compute $\displaystyle\frac{\dd \ln p(d_n\given\theta)}{\dd\theta}$
    \end{itemize}
  \item reduce:
    \begin{itemize}
    \item pairwise sum
    \end{itemize}
  \item Go uphill.  Repeat as necessary; each iteration only takes
    $\log N$ time.
    \begin{itemize}
    \item (use L-BFGS or conjugate-gradient or whatever you like)
    \item complexity is $N\,\log N$, \emph{compute time} is $\log N$
    \end{itemize}
  \item Total optimization typically takes $M^2\log N$ time, where $M$ is the \emph{number of parameters}.
    \begin{itemize}
    \item ($M^3\log N$ if you want a full error analysis)
    \end{itemize}
  \end{itemize}
\end{frame}

\begin{frame}
  \frametitle{all that matters is the number of parameters}
  \begin{itemize}
  \item So we are done, right?
  \item<2-> \emph{Nope.}
  \item<3-> In all \emph{real} problems, the number of parameters $M$ \emph{scales with the data volume $N$}.
  \item<3-> If you want to go beyond maximum-likelihood, things get \emph{hard}.
  \end{itemize}
\end{frame}

\begin{frame}
  \frametitle{physics problems are hierarchical}
  \includegraphics{../pgm/weaklensing.pdf}
\end{frame}

\begin{frame}
  \frametitle{nuisance parameters}
  \begin{itemize}
  \item Nuisance parameters tend to increase in number with the data size.
  \item Relevant backgrounds get more subtle and must be modeled more carefully.
  \item Details of sample selection and observed data distribution functions become more important.
  \item As precision expectations rise (and they rise with $N$), noise models get more realistic.
    \begin{itemize}
    \item all these effects bring new parameters
    \end{itemize}
  \end{itemize}
\end{frame}

\begin{frame}
  \frametitle{calibration parameters}
  \begin{itemize}
  \item As you go from $N=10^3$ to $N=10^9$, you expect to do more than $10^3$ times better in accuracy.
  \item Calibration of the device must get correspondingly better.
    \begin{itemize}
    \item time-dependence, temperature, Solar Cycle, hysteresis
    \item all these effects bring new parameters
    \end{itemize}
  \end{itemize}
\end{frame}

\begin{frame}
  \frametitle{Bayesian inference \emph{isn't} map--reduce}
  \begin{itemize}
  \item $\displaystyle p(\theta\given D) = \frac{1}{Z}\,p(\theta)\,\prod_n p(d_n\given\theta)$
  \item map:
    \begin{itemize}
    \item compute functions $p(d_n\given\theta)$
    \end{itemize}
  \item reduce:
    \begin{itemize}
    \item product functions together (starting with the prior)
    \end{itemize}
  \item but think about how you pass forward those functions
    \begin{itemize}
    \item $\theta$ has $10^6$ or more parameters
    \item functions are multi-modal
    \item support is broader than Gaussian
    \item when the data get large, the \emph{resolution required} becomes unsustainable
    \end{itemize}
  \end{itemize}
\end{frame}

\begin{frame}
  \frametitle{even the frequentists are doomed}
  \begin{itemize}
  \item All the ``$M$ scaling with $N$'' arguments apply to frequentists and Bayesians alike.
  \item Computing the full-data likelihood \emph{function} is just as hard as computing the full-data posterior PDF.
  \end{itemize}
\end{frame}

\begin{frame}
  \frametitle{marginalization is hard---and unavoidable}
  \includegraphics{../pgm/weaklensing.pdf}
\end{frame}

\begin{frame}
  \frametitle{Bayesian state-of-the-art}
  \begin{itemize}
  \item There \emph{are} huge non-parametric Bayesian inferences with
    massive marginalizations out there.
  \item How were they done?
    \begin{itemize}
    \item carefully chosen priors that make the inferences and
      marginalizations analytic or tractable
    \item \emph{we can't do this}
    \item why not?  Because for us the priors \emph{actually are} our
      prior beliefs.  Our \emph{real} prior beliefs are not conjugate
      to anything!
    \end{itemize}
  \item ``Bayesian'' is becoming a bad word.
  \end{itemize}
\end{frame}

\begin{frame}
  \frametitle{my approach}
  \begin{itemize}
  \item<2-> Brute Force (tm).
  \item<3-> Plus some help from applied math and computer vision.
    \begin{itemize}
    \item approximate Bayesian inference
    \item very clever Markov-Chain Monte Carlo methods
    \item exploiting problem sparsity
    \end{itemize}
  \end{itemize}
\end{frame}

\conclusion

\end{document}

  \item \emph{non-parametric} shear map or distribution function
    \begin{itemize}
    \item ``non-parametric'' means the model \emph{gets bigger as the
      data set gets bigger} (or better)
    \item think: \emph{As you observe more and more galaxies, with
      better redshift estimates, you increase the angular and redshift
      resolution of your shear map.}
    \item importantly, non-parametric models are \emph{never} inferred
      at high signal-to-noise
    \end{itemize}

\begin{frame}
  \frametitle{there are no linear problems}
  \begin{itemize}
  \item Even if your noise is Gaussian, you never know the noise
    variance at high precision.
  \item In most real situations, the data are produced by a
    \emph{mixture of processes}.
  \item There are always multiple modes to the likelihood function,
    and broad support in parameter space.
  \end{itemize}
\end{frame}

