% Copyright 2012 David W. Hogg (NYU).  All rights reserved.

\documentclass[pdftex]{beamer}
\usepackage{amssymb,amsmath,mathrsfs}
\usecolortheme{default}

% this one is debatable
\renewcommand{\emph}[1]{\textbf{#1}}

%%% color commands
\newcommand{\whiteonblack}{%
  \colorlet{fg}{white}
  \colorlet{bg}{black}
  \setbeamercolor{normal_text}{fg=white,bg=black}
  \setbeamercolor{background canvas}{fg=white,bg=black}
  \setbeamercolor{alerted_text}{fg=yellow}
  \setbeamercolor{example_text}{fg=white}
  \setbeamercolor{structure}{fg=white}
  \setbeamercolor{palette_quaternary}{fg=white}
}
\newcommand{\blackonwhite}{%
  \colorlet{fg}{black}
  \colorlet{bg}{white}
  \setbeamercolor{normal_text}{fg=black,bg=white}
  \setbeamercolor{background canvas}{fg=black,bg=white}
  \setbeamercolor{alerted_text}{fg=blue}
  \setbeamercolor{example_text}{fg=black}
  \setbeamercolor{structure}{fg=black}
  \setbeamercolor{palette_quaternary}{fg=black}
}
\xdefinecolor{pink}{rgb}{1.0,0.9,0.9}

%%% size and shape commands
\newlength{\figurewidth}
\setlength{\figurewidth}{0.9\textwidth}
\newlength{\figureheight}
\setlength{\figureheight}{0.9\textheight}

%%% text commands
\newcommand{\project}[1]{\textsl{#1}}
  \newcommand{\an}{\project{Astrometry.net}}
  \newcommand{\euclid}{\project{Euclid}}
  \newcommand{\flickr}{\project{flickr}}
  \newcommand{\gaia}{\project{Gaia}}
  \newcommand{\galex}{\project{GALEX}}
  \newcommand{\kepler}{\project{Kepler}}
  \newcommand{\GALEX}{\galex}
  \newcommand{\hst}{\project{HST}}
  \newcommand{\hipparcos}{\project{Hipparcos}}
  \newcommand{\lsst}{\project{LSST}}
  \newcommand{\sdss}{\project{SDSS}}
  \newcommand{\sdssiii}{\project{SDSS-III}}
  \newcommand{\sdssiv}{\project{SDSS-IV}}
  \newcommand{\boss}{\sdssiii\ \project{BOSS}}
  \newcommand{\osss}{\project{OSSS}}
  \newcommand{\ska}{\project{SKA}}
  \newcommand{\vo}{\project{VO}}
  \newcommand{\rttd}{\project{Right Thing To Do}$^{\mbox{\scriptsize\sffamily{TM}}}$}
\newcommand{\foreign}[1]{\textit{#1}}
\newcommand{\latin}[1]{\foreign{#1}}
  \newcommand{\cf}{\latin{cf.}}
  \newcommand{\eg}{\latin{e.g.}}
  \newcommand{\etal}{\latin{et~al.}}
  \newcommand{\etc}{\latin{etc.}}
  \newcommand{\ie}{\latin{i.e.}}
  \newcommand{\vs}{\latin{vs.}}

%%% math-mode commands
\newcommand{\unit}[1]{\mathrm{#1}}
  \newcommand{\rad}{\unit{rad}}
  \newcommand{\s}{\unit{s}}
  \newcommand{\yr}{\unit{yr}}
  \newcommand{\km}{\unit{km}}
  \newcommand{\kmps}{\km\,\s^{-1}}
\newcommand{\mmatrix}[1]{\boldsymbol{#1}}
\newcommand{\tv}[1]{\boldsymbol{#1}}
\newcommand{\dd}{\mathrm{d}}
\newcommand{\given}{\,|\,}
 % hogg standard colors
\setlength{\paperheight}{3.5in}
% 1.77778 is the ratio of 16 to 9
\setlength{\paperwidth}{1.77778\paperheight}
\setlength{\textwidth}{0.85\paperwidth}
\usepackage{amssymb,amsmath,mathrsfs}

\title{1.~hierarchical probabilistic inference \\ {\small and} \\ 2.~the costs and benefits of sharing data}
\author[David W. Hogg (NYU)]{David W. Hogg \\
  \textsl{\small Center for Cosmology and Particle Physics,
                 New York University}}
\date{2012 November 29}

\begin{document}

\begin{frame}
  \titlepage
\end{frame}

\begin{frame}
  \frametitle{infer dynamics from kinematics {\small (0903.5308)}}
  \begin{itemize}
  \item \project{Gaia} will give a snapshot of positions and
    velocities for $10^7$ to $10^9$ stars.  How do we figure out the
    gravitational potential of the Galaxy?
  \item<2-> Let's start by doing this in the Solar System.
  \item<2-> If we can't do the \emph{Solar System} we can't do anything!
  \item<2-> Imagine that you had a \emph{snapshot} of the planet positions
    and velocities on 2009~April~1.
  \item<2-> Could you infer that the force law is $1/r^2$~?
  \end{itemize}
\end{frame}

\begin{frame}
  \frametitle{infer dynamics from kinematics {\small (0903.5308)}}
  \begin{itemize}
  \item In steady-state, $f(\tv{x},\tv{v})$ is a function of conserved
    quantities only.
  \item $\displaystyle
        p(\tv{x}_i,\tv{v}_i|\tv{\omega},\tv{\alpha})
      = \left|\left|\frac{\dd\tv{I}\,\dd\tv{\phi}}{\dd\tv{x}\,\dd\tv{v}}\right|\right|_{\tv{\omega}}\,\left[\frac{1}{2\pi}\right]^3
      \,p(\tv{I}|\tv{\alpha})$
  \item<2-> $\displaystyle
        p(\tv{x}_i,\tv{v}_i|\tv{\omega})
      = \int \dd\tv{\alpha}\,p(\tv{\alpha})
      \,p(\tv{x}_i,\tv{v}_i|\tv{\omega},\tv{\alpha})$
  \item<2-> Marginalization is hard:
    \begin{itemize}
    \item 200 parameters in the marginalization
    \item more parameters than data!
    \item priors from Gaussian processes
    \end{itemize}
  \end{itemize}
\end{frame}

\begin{frame}
  \frametitle{infer dynamics from kinematics {\small (0903.5308)}}
  \includegraphics[width=\textheight]{../../pgm/gaia.pdf}
\end{frame}

\newlength{\jacwidth}
\setlength{\jacwidth}{0.25\textwidth}

\begin{frame}
  \frametitle{infer dynamics from kinematics {\small (0903.5308)}}
  \includegraphics[width=\jacwidth]{jacobian-3.png}\includegraphics[width=\jacwidth]{jacobian-7.png}\includegraphics[width=\jacwidth]{jacobian-0.png}\includegraphics[width=\jacwidth]{jacobian-2.png} \\
  \includegraphics[width=\jacwidth]{jacobian-1.png}\includegraphics[width=\jacwidth]{jacobian-5.png}\includegraphics[width=\jacwidth]{jacobian-6.png}\includegraphics[width=\jacwidth]{jacobian-4.png}
\end{frame}

\begin{frame}
  \frametitle{infer dynamics from kinematics {\small (0903.5308)}}
  \includegraphics[width=\textheight]{alpha.png}
\end{frame}

\begin{frame}
  \frametitle{infer dynamics from kinematics {\small (0903.5308)}}
  \includegraphics[width=\textheight]{virial.png}
\end{frame}

\begin{frame}
  \frametitle{infer dynamics from kinematics {\small (0903.5308)}}
  \begin{itemize}
  \item The phase-space DF model is so general, it can \emph{discover}
    phase-space structure.
  \item There \emph{is} phase-space structure.
  \item All currently used point estimates---even maximum-likelihood
    ones---either have this hard-coded (bad) or can't discover it
    (bad).
  \item (That said, the procedure was \emph{expensive}.)
  \end{itemize}
\end{frame}

\begin{frame}
  \frametitle{models of no fixed complexity {\small (1211.5805)}}
  \includegraphics[width=\textwidth]{test_cases_copy.pdf}
\end{frame}

\begin{frame}
  \frametitle{models of no fixed complexity {\small (1211.5805)}}
  \begin{itemize}
  \item Images filled with stars; want to know the luminosity function of stars.
  \item Catalog \emph{itself} is a very large blob of nuisance parameters.
  \item Don't know (and don't care) precisely how many stars there are.
  \item Faint stars contribute to the noise.
  \item connections to LIGO:
    \begin{itemize}
    \item ``faint stars'' could be fainter GW sources
    \item ``faint stars'' could be seismic events
    \end{itemize}
  \end{itemize}
\end{frame}

\begin{frame}
  \frametitle{models of no fixed complexity {\small (1211.5805)}}
  [NEED PGM HERE]
\end{frame}

\begin{frame}
  \frametitle{models of no fixed complexity {\small (1211.5805)}}
  \includegraphics[width=\textwidth]{catalogs_copy.pdf}
\end{frame}

\begin{frame}
  \frametitle{models of no fixed complexity {\small (1211.5805)}}
  \includegraphics[height=0.45\textheight]{broken_copy.pdf}\\
  \includegraphics[height=0.45\textheight]{inference2_copy.pdf}
\end{frame}

\begin{frame}
  \frametitle{latent time-domain models {\small Listgarten \etal}}
  [SOUND FIGURE HERE]
\end{frame}

\begin{frame}
  \frametitle{latent time-domain models {\small Listgarten \etal}}
  [CHROMATOGRAPHY FIGURE HERE]
\end{frame}

\begin{frame}
  \frametitle{latent time-domain models {\small Listgarten \etal}}
  \begin{itemize}
  \item There is a latent ``true'' time-domain function.
  \item There is a set of free ``time warping'' functions.
  \item More parameters than data (truly non-parametric).
  \item connections to LIGO:
    \begin{itemize}
    \item small wrongnesses in template can make template orthogonal to signal
    \item very good ideas about priors or regularization
    \end{itemize}
  \end{itemize}
\end{frame}

\begin{frame}
  \frametitle{conclusions 1: hierarchical inference}
  \begin{itemize}
  \item Hierarchical inference makes experiments more accurate and precise.
    \begin{itemize}
    \item ``Data-driven regularization''.
    \item The prior for each data point is informed by all other data points.
    \item Often it is only the \emph{hyperparameters} that you care about.
    \end{itemize}
  \item Expect to spend a huge number of parameters on nuisances.
    \begin{itemize}
    \item Don't be afraid to have more parameters than data.
    \item Don't like priors?  You only need them on your nuisance parameters.
    \end{itemize}
  \end{itemize}
\end{frame}

\begin{frame}
  \frametitle{the costs of sharing data}
\end{frame}

\begin{frame}
  \frametitle{Sloan Digital Sky Survey}
\end{frame}

\begin{frame}
  \frametitle{Hipparcos}
\end{frame}

\begin{frame}
  \frametitle{Wilkinson Microwave Anisotropy Probe}
\end{frame}

\begin{frame}
  \frametitle{benefits of sharing data}
\end{frame}

\begin{frame}
  \frametitle{functional testing}
\end{frame}

\begin{frame}
  \frametitle{what to release}
\end{frame}

\begin{frame}
  \frametitle{licensing}
\end{frame}

\begin{frame}
  \frametitle{conclusions 2: sharing data}
\end{frame}

\end{document}
